\chapter{Methodik der Datenerhebung}
\label{ch:methoden}

\section%
	[Datenerhebung aus dem \tit{Corpus der altdeutschen Originalurkunden}]%
	{Datenerhebung aus dem \tit{Corpus der altdeutschen\\ Originalurkunden}}
\label{sec:miningcao}

Die Daten für die Teilauswertung zum \citetitle*{cao} (CAO) wurden aus der in
\textcites[207]{beckerschallert2021}[155--158]{beckerschallert2022b}
beschriebenen Datenbank exzerpiert. Grundlage dafür bildet die digitalisierte
Fassung des \CAO{}
\autocites{cao-online}[vgl.~dazu][]{gniffkerapp2005} inklusive der von
\citeauthor{beckerschallert2022b} vorgenommenen Zuordnung von Ortskoordinaten.
Das \CAO{} umfasst insgesamt etwa zwei Millionen Wortformen inklusive
Mehrfachausfertigungen und nicht-deutschsprachiger Urkunden. Pro Urkunde liegen
damit durchschnittlich 453 Wortformen vor, die Standardabweichung beträgt 500
Wortformen.

Wie in \cref{sec:materialcao} bezüglich der Text\-treue des \CAO{}
beschrieben, gehe ich davon aus, dass die in der Edition gebotenen
Transkriptionen für die Zwecke dieser Untersuchung hinreichend fehlerfrei sind.
Bei der Belegsammlung wurden nur solche Urkunden berücksichtigt, die eine
eindeutige Jahresangabe besitzen und die laut der Zuordung im
Schreibortverzeichnis von \citet{cao-online} einem einzigen Ausstellungsort
zugeordnet sind. Bei der näheren Beschäftigung mit dem Material hat sich gezeigt, dass der Zeitfaktor weniger ausschlaggebend ist, da
ohnehin die meisten Textstücke im \CAO{} aus der Zeit zwischen 1280
und 1299 stammen (vgl.~\cref{tab:urkstat}).

Bezüglich des Orts sollen aber nur Ausstellungs\-kontexte mit möglichst
kleinräumigem Bezug erfasst werden, um Urkundentexte mit überregionaler Geltung
und gegebenenfalls damit einhergehender Vermeidung von Regionalismen nach
Möglichkeit auszuschließen. Implizit lehnen sich diese Kriterien an die in
\citet[41--42]{ganslmayer2012} diskutierten an, wobei der Faktor \q*{Text} in
Bezug auf Parallelausfertigungen und zeitferne Abschriften ignoriert wurde.
Wenn durch die in \citet[155--158]{beckerschallert2022b} geschilderten
Ausschlusskriterien lediglich etwa 50\pct\ des Gesamttextbestands zur Verfügung
stehen, dürften die 8,9\pct\ der deutschsprachigen Urkunden, die
\citeauthor{ganslmayer2012} aufgrund des Faktors \q*{Text} ausschließt, kaum
ins Gewicht fallen.%
%
	\footnote{Neben 4.289 deutschsprachigen Urkunden weist
		\citet[41]{ganslmayer2012} insgesamt 35 lateinische und 293
		niederländische Urkunden aus.}
%
Des Weiteren wurde auf eine Bestimmung des Schreibdialekts jeder einzelnen
ausgewerteten Urkunde im Abgleich mit ihrem angegebenen Ausstellungsort
verzichtet.

Die Suche in der Datenbank geschah
% , wie in
% \textcites[207--209]{beckerschallert2021}[155--158]{beckerschallert2022b}
% beschrieben,
mit Hilfe
% des in \cref{ex:regexcao} aufgeführten
eines regulären Ausdrucks \autocite[dazu z.\,B.][33--37]{perkuhnetal2012}, der
auf Basis der Liste der im \CAO{} belegten Formen des Lemmas
\norm{bėide} formuliert wurde \autocites(mit allen Deklinationsformen insgesamt
ca.~2.050 Belege)[vgl.][166--168]{wmu1}. \cref{tab:beidespelcao} listet die
belegten grafischen Varianten pro Position im Wort entsprechend diesem
Suchausdruck tabellarisch auf.%
%
	\footnote{Die einzige im \tit{\WMU{}} verzeichnete Form mit
		\lit{-t-} ist \lit{beyte} \autocite[166]{wmu1}: \lit{durch beyte deſ
		vor genanten Hugeſ von Luzelſteyn vn̄ Henriches von Fleckenſteyn}
		\wdef{durch beide, des vorgenannten Hugo von Lützelstein und Heinrichs
		von Fleckenstein} \autocites(Straßburg, 1294)[\pno~N~674,
		484.18]{cao5}. Diese Urkunde war nicht Teil der Auswertung. Formen wie
		\lit{bete} oder \lit{pet} stehen ansonsten für \norm{bęte} \wdef{Bitte,
		Gebet} oder stellen Formen von \norm{biten} \wdef{bitten} dar.}
%
Bei der Suche nach Zeichenketten wurden die Belegstellen nicht automatisch nach
dem syntaktischen Funktionskontext des jeweiligen Belegs (determinierender
Quantor, Konjunktion) geschieden. Diese Zuordnung geschah manuell in einem
zweiten Schritt.

% \needspace{3\baselineskip}
% \begin{exe}
% \ex \begin{verbatim}
% \b(?:
%   (?:b(?:[eêéæ]|e[iîye]|eͤ|êi|a[iîy]|aͤ|aͥ|âi|æi|i[æe])
%     |p(?:[ea][iy]|[eæ]|æi)
%   )[dt]
%   (?:eiw|uͥ|vͥ|u̍|v̍|ú|v́|iͮ|j|i(?:û|v̂|uͦ|vͦ|uͤ|vͤ)|[ie][uvw]?|[uv])?
% )\b
% \end{verbatim}
% \label{ex:regexcao}
% \end{exe}

\begin{table}
\centering
\caption{Im \WMU{} belegte Schreibweisen von mhd.~\norm{bėide/-iu} pro
	Position im Wort}
\begin{tabular}{c c c c}
\toprule

\mc{3}{c}{\bfseries Stamm}
	& \bfseries Flexion
	\\

\cmidrule(r){1-3}
\cmidrule(l){4-4}

% \begin{minipage}{1em}
\makecell{
	b\\
	p
}
% \end{minipage}
	& \begin{minipage}{6em}
		\begin{multicols}{4}\centering
		ai\\
		ay\\
		aî\\
		aͤ\\
		aͥ\\
		ee\\
		ei\\
		ey\\
		eî\\
		eͤ\\
		ie\\
		iæ\\
		âi\\
		æi\\
		êi\\
		e\\
		æ\\
		é\\
		ê
		\end{multicols}
	\end{minipage}
	& \makecell{
			d\\
			(t)
		}
	& \begin{minipage}{6em}
		\begin{multicols}{4}\centering
			eiw\\
			iuͦ\\
			ivͤ\\
			iͤv\\
			eu\\
			ev\\
			ew\\
			iu\\
			iv\\
			iû\\
			iͮ\\
			uͥ\\
			vͥ\\
			îv\\
			e\\
			i\\
			j\\
			u\\
			v\\
			ú\\
			û\\
			Ø
		\end{multicols}
	\end{minipage}
	\\

\addlinespace[1ex]
\bottomrule
\end{tabular}
\label{tab:beidespelcao}
\end{table}

\phantomsection
\label{phsec:caohiatus}
Insgesamt wurden 401 Belege zu \norm{bėide}-Targets erfasst, von denen 244 auf
Kontexte mit Quantor entfielen, 157 auf Kontexte mit Konjunktion; 55
Belegstellen wurden aus verschiedenen Gründen ausgeschlossen
(vgl.~\cref{tab:ausgewcao,sec:urkliste}). Von den 4.289 im \CAO{}
enthaltenen Urkunden und 1.021 zuordenbaren Ausstellungsorten sind in der
Stichprobe 291 Urkunden aus 114 Orten vertreten, von denen 256 Urkunden aus 102
Orten für die Auswertung verwendbar waren.

\begin{table}
\centering
\caption{Zahl der gesammelten und ausgewerteten Belegstellen}
\begin{tabular}{l r r r}
\toprule

%
	& \bfseries Gesammelt
	& \bfseries Ausgewertet
	& \bfseries in \pct
	\\

\midrule

% Zahlen retrospektiv überprüft am 01.08.23 an
% data/2022-05-19_beide_cao_controller-target.ods

Quantor
	& 244 % gesamt
	& 219 % ausgewertet
	& 89,8 % in %
	\\

Konjunktion
	& 157 % gesamt
	& 127 % ausgewertet
	& 80,9 % in %
	\\

\midrule

Summe
	& 401 % gesamt
	& 346 % ausgewertet
	& 86,3 % in %
	\\

\bottomrule
\end{tabular}
\label{tab:ausgewcao}
\end{table}

Zu allen Kongruenztargets wurden die Controller erfasst -- im Fall von Targets
in unmittelbarer Abhängigkeit von pronominalen Controllern diese sowie die
letzte Nennung der Referenten, auf die sie verweisen (\q*{Erstcontroller}).
Alle Controller und Targets wurden anschließend nach ihren formalen und
semantischen Personenmerkmalen annotiert. Außerdem wurde jeweils die Wortart
festgehalten sowie die Flexionsform des Targets mit deren Flexionstyp
(\norm{-e}, -Ø, \norm{-iu}) und ob gegebenenfalls Schwa-Apokope vorliegt. Auch
der lineare Abstand der betreffenden aufeinander bezogenen Wortformen und die
Domäne der Kongruenzbeziehung (gleiche Phrase, gleicher Teilsatz, anderer Satz)
wurden registriert.

Targets, bei denen in Urkunden vom gleichen Ort nach kursorischer Durchsicht
kein Unterschied in der starken Adjektivdeklination zwischen \norm{-e} und
\norm{-iu} im Nom./Akk.~Pl.\ ausgemacht werden konnte, wurden ausgesondert
(vgl.~\cref{sec:adjdeclcao}). Targets, bei denen der Vokal des Flexionssuffixes
im Hiatus steht und der daher zumindest theoretisch elidierbar ist
\autocites[vgl.][90--91]{askedal1973}[191]{gjelsten1980} wurden gesondert
markiert, wenn auch die Urkundentexte nicht an ein metrisches Schema gebunden
sind.

% \subsection{Ausgeschlossene Belege}
% \label{subsec:ausgeschlossene_cao}

% Neben diesen dialektalen Ausschlusskriterien wurde ein Beleg aufgrund von
% inhaltlichen Ambiguitäten ausgeschlossen \cref{ex:beidesamt}. Dieser ist an
% sich einschlägig, doch gibt es hier Komplikationen bei der Referenz von
% \lit{beidú ſamit} \wdef{alle beide}. Zunächst könnte man vermuten, dass sich
% dieser Ausdruck auf \lit{die bruͦdir von ſant Johanniſ}
% \wdef{die Brüder von St.~Johannes} bezieht, wobei diese nicht persönlich
% erwähnt werden.

% \begin{exe}
% \ex\label{ex:beidesamt}
% 	\lit{Vnde hant die brůdir von ſant Johanniſ daz ſelbi guͦt / \textbf{beidú}
% 	ſamit / wider verlúhen Heinrich jrme ſun} \\
% 	\begin{taggedline}{\parencites(Freiburg i.\,Br., 1273)[\pno~201,~211.25--26]{cao1}}
%     	\wdef{Und die Brüder von St.~Johannes haben dasselbe Gut, alle beide,
%     	wieder verliehen an Heinrich, ihren \textins{=~Frau Agnes'} Sohn.}
% 	\end{taggedline}
% \end{exe}

% Andererseits ist der Verhandlungsgegenstand der Urkunde Frau Agnes' \lit{hûſ
% ze Adilar / vnde daz vierteil jrſ gvͦteſ ze \smash{Oͤliſwiller}} \wdef{Haus
% in Adilar und das Viertel ihres Gutes in Öliswiller}
% \autocites(N.~Sg~+~N.~Sg.)[\pno~201,~211.23]{cao1}, wobei im weiteren Text
% stets nur von \norm{\textins{dem} ſelbi\textins*{n} guͦt} \wdef{demselben
% Gut} (N.~Sg.) die Rede ist. Es scheint dabei unlogisch, dass sich \lit{beidú
% ſamit} \wdef{alle beide} als Ausdruck der betonten Zweiheit nur auf \lit{daz
% vierteil jrſ gvͦteſ} \wdef{das Viertel ihres Gutes} beziehen sollte; genauso
% wird es mehr als zwei Brüder im Johanniterstift gegeben haben. Sinnvoller
% wäre, \lit{guͦt} \wdef{Gut} in \cref{ex:beidesamt} verallgemeinert im Sinne
% von \wdef{Besitztum} zu interpretieren. \lit{\textins{B}eidú ſamit} wäre
% damit als Apposition zu lesen, die unterstreicht, dass mit \lit{guͦt} beide
% Gegenstände gemeint sind: Haus und Viertel des Gutes. Der Controller von
% \lit{beidú} ist daher syntaktisch trotz allem \lit{guͦt}, allerdings
% semantisch umfassend das \lit{hûſ} und das
% \lit{vierteil}.

%%%%%%%%%%%%%%%%%%%%%%%%%%%%%%%%%%%%%%%%%%%%%%%%%%%%%%%%%%%%%%%%%%%%%%%%%%%%%%%

\section{Datenerhebung aus der \tit{Kaiserchronik}}

Die Texte der \KC{} (KC) liegen dank \citetitle{kcdigital}
\autocite{kcdigital} im XML-Format vor. Dieses kann problemlos in Klartext
zurückgewandelt werden, indem Annotationstags verworfen oder serialisiert
werden.%
%
	\footnote{Ich danke Magnus Breder Birkenes (Oslo), der diesen Arbeitsschritt
	bereits vollzogen und mir die Klartexte sämtlicher Textzeugen zur Verfügung
	gestellt hat.}
%
Dies ermöglicht wie auch beim \CAO{}-Material eine Textsuche mit Hilfe
von regulären Ausdrücken. Die Vorgehensweise gestaltet sich dennoch leicht
anders, da für die \KC{} kein lemmatisiertes Verzeichnis von grafischen
Varianten wie mit dem \WMU{} für die Urkunden vorliegt, sodass kein
maßgeschneiderter Suchausdruck aufgrund bereits indizierter Schreibvarianten
destilliert werden kann. Stattdessen ist es notwendig, sich an die vertretenen
Schreibweisen durch begründete Annahmen heranzutasten. Aus der Arbeit am
\CAO{} ist zu schließen, dass das Konsonantengerüst des Wortstamms
regelmäßig die Form \textit{b/p--d} hat; auch die Lemmaliste zum
\tit{Mittelhochdeutschen Wörterbuch} von \citeauthor{mwb1}
\autocite[\pno~\fw{beide}]{mwb1} enthält diesbezüglich keine anderweitige
Evidenz.%
%
	\footnote{Siehe
		\url{http://www.mhdwb-online.de/konkordanz.php?lid=13638000&seite=1}
		(\DTMdate{2021-12-28}). Die grafische Variation unter den 2.559 Belegen,
		die zum gegenwärtigen Zeitpunkt aufgelistet werden, bemisst sich auf
		\lit{b, p} -- \lit{ai, ay, ei, ey, eí, eǐ, éi, êi, é, ê} -- \lit{d} --
		\lit{iͤv, eu, ev, iu, iv, uͦ, íu, iͮ, a, e, i, u, è, ú, û,} Ø. In den
		ausgewerteten Handschriften der \KC{} gibt es keine Anzeichen für
		Schreibvarianten von \norm{bėide} mit \lit{-t-}, auch nicht in
		Kontexten mit Apokope. Formen ähnlich \norm{bėide} mit \lit{-t-} oder
		\lit{-t} sind den Lemmata \norm{bęte} \wdef{Bitte, Gebet},
		\norm{bęten/biten} \wdef{bitten, beten}, \norm{bėtte} \wdef{Bett},
		\norm{bėiten} \wdef{zögern, warten} und \norm{bieten} \wdef{(an)bieten}
		zuzuordnen.}
%
Für die gegenwärtige Auswertung kommen nur die Flexionsformen vom Typ \norm{-e}
und \norm{-iu} in Frage, daher wurden bei der Belegsammlung Formen, bei denen
in der Zeichenkette nach dem Wortstamm \norm{m}, \norm{n}, \norm{r}, \norm{s}
oder \norm{z} enthalten sind, nicht berücksichtigt; Falschpositive wurden
manuell überprüft und eliminiert. \cref{tab:beidespelkc} gibt eine Übersicht
über die für das \KC{}-Material ermittelten grafischen Varianten der
Wortformen \norm{bėide} und \norm{bėidiu} pro Position im Wort. Eine
hypothetische Form \fw{*weid-} mit bairischer Verwechslung von \norm{b} und
\norm{w} \autocite[153]{paul2007} konnte nicht belegt werden.

% \b(?:
%   b(?:
%     aedev|æidiv|a[yÿ]de|ejde|eyde?|êide|aid(?:
%       e[uv]?|[ií]v|u̍
%     )?|aíd(?:
%       [eí]v|u̍|e
%     )|eid(?:
%       e[uv]|iv|í[uv]|[ev]
%     )?|evdi[uv]|eíd(?:
%       í[uv]|e
%     )|ed(?:
%       [ií][uv]|e
%     )?|æd(?:
%       [ei]v|e
%     )
%   )
%   |p(?:
%     aideu|aẏ?dív|a[yÿ]de|ede?|aid(?:
%       i[uv]|ív|e
%     )?|aíd(?:
%       [ií]v|e
%     )|eid(?:
%       eu?|[ií]v
%     )?|eyd(?:
%       iv|e
%     )
%   )
% )\b

\begin{table}
\centering
\caption{In der \KC{} belegte Schreibweisen von mhd.~\norm{bėide/-iu} pro
	Position im Wort}
\begin{tabular}{c c c c}
\toprule

\mc{3}{c}{\bfseries Stamm}
	& \bfseries Flexion
	\\

\cmidrule(r){1-3}
\cmidrule(l){4-4}

% \begin{minipage}{2em}
\makecell{
	b\\
	p
}
% \end{minipage}
	& \begin{minipage}{6em}
		\begin{multicols}{4}\centering
		ae\\
		ai\\
		ay\\
		aí\\
		aÿ\\
		aẏ\\
		ei\\
		ej\\
		ev\\
		ey\\
		eí\\
		æi\\
		êi\\
		a\\
		e\\
		æ
		\end{multicols}
	\end{minipage}
	& d
	& \begin{minipage}{4em}
		\begin{multicols}{3}\centering
			eu\\
			ev\\
			iu\\
			iv\\
			íu\\
			ív\\
			u̍\\
			e\\
			Ø
		\end{multicols}
	\end{minipage}
	\\

\addlinespace[1ex]
\bottomrule
\end{tabular}
\label{tab:beidespelkc}
\end{table}

Die Auswahl der Textzeugen beschränkt sich auf die drei Leithandschriften der
neuen \KC{}-Edition von \citeauthor{chincaetal2019b}
(Siglen \cite{kc:A1}, \cite{kc:B1}, \cite{kc:C1}) sowie die Vergleichs- (\cite{kc:H}, \cite{kc:VB})
und Parallelhandschriften, insofern sie umfassende Teile des Texts enthalten
(\cite{kc:M}, \cite{kc:P}, \cite{kc:K}, \cite{kc:Z}).%
%
	\footnote{Im Folgenden werden die einzelnen Textzeugen der \KC{} der
	Kürze halber mit Siglen bezeichnet. Diese richten sich nach
	\citetitle{kcdigital} \autocite{kcdigital}.}
%
Ausgenommen wurden die Vergleichshandschrift \citet{kc:VC} sowie die
Parallelhandschrift \citet{kc:W}: bei Vorarbeiten hatte sich \citet{kc:VC}
bezüglich ihres Sprachstands als problematisch erwiesen, da der Text im
bairisch-österreichischen Schreibdialekt (mit leichtem mitteldeutschen
Einschlag) \blockcquote[73]{wolf:kckat}{deutlichen Änderungen im Laufe des
Schreibprozesses} unterliegt. \citet{kc:W} stellt eine Mischredaktion aus A und
C dar und enthält außerdem die \tit{Prosa\-kaiserchronik}
\autocite[48--54]{weis2022}, ist also ebenfalls ein sprachlich heterogener
Text. Fragmente wurden bei der Auswertung generell nicht berücksichtigt, da sie
im Durchschnitt einen einzigen Beleg für \norm{bėide} pro Textzeuge liefern,
jedoch in 14 von 38 Fällen keinen einzigen.

Neben einzelnen Episoden mit weiblichen Protagonisten -- zum Beispiel Veronica,
Lucretia oder Crescentia \autocite[729--838, 4335--4772,
11518--12808]{schroeder1895} -- handelt die \KC{} dem Prolog nach
vornehmlich

\blockcquote[][\pno~19--20]{schroeder1895}[]{
	\norm{von den bâbesen unt von den chunigen,\\
	baidiu guoten unt ubelen}

	\q*{von den Päpsten und von den Königen,\\
	sowohl guten als auch schlechten}
}

Aufgrund der thematischen Ausrichtung geht es also vornehmlich um Männer. Paare
aus Mann und Frau kommen extrem selten vor. Die \KC{} ist trotz allem
eine Untersuchung wert, da es sich bei ihr um einen der prestige- und
einflussreichsten Texte des deutschsprachigen Mittelalters handelt
\autocite[93]{wolf2008}, durch dessen digitale Erschließung mit zeichengenauer
Transkription \autocite{kcdigital} der Sprachwissenschaft ein umfangreiches
Textkorpus aus einem Guss zur Verfügung steht. Die große Menge der Textzeugen
auch innerhalb der einzelnen Rezensionen macht die \KC{} attraktiv als
Parallelkorpus \autocite{cysouwwaelchli2007}. Alle 53 derzeit über
\citetitle{kcdigital} verfügbaren Textstücke aus 47 verschiedenen Handschriften
umfassen etwa 995.832 Wortformen, wobei sich ein Mittelwert pro Handschrift von
21.649 Wortformen ergibt; die Standardabweichung liegt bei 34.652 Wortformen.

Von 59 gesammelten Belegen mit kombiniertem gemischtgeschlechtlichen Bezug
liegen lediglich sechs attributiv gegenüber 53 mit \norm{bėide} als Teil der
korrelativen Konjunktion \norm{bėide \dots\ unde} \wdef{sowohl \dots\ als auch}
vor. Dass nur sehr wenige attributive Belege in der Stichprobe enthalten sind,
macht den Vergleich zu den Urkunden des \CAO{} besonders interessant,
da dort umgekehrte Mengenverhältnisse vorliegen: Im \CAO{} stehen 76
Belege mit kombiniertem gemischtgeschlechtlichen Bezug zur Verfügung, von denen
70 attributiv sind und lediglich sechs Teil der korrelativen Konjunktion. Die
Belegsammlung zur \KC{} lässt sich so als Gegengewicht zu derjenigen des
\CAO{} verstehen, nicht nur, was die Textgattung betrifft. Die
\cref{tab:beidevar} schlüsselt die Zahl der gesammelten Belege nach Handschrift
mit ihrer jeweiligen Sigle, Funktion und Wortform auf.

\begin{table}
\centering
\caption{Vorkommen von \norm{bėide/-iu} in den exzerpierten Handschriften nach Funktion}
% Diese Tabelle enthält *alle* exzerpierten Belege, ungefiltert!
\begin{tabular}[t]{
	l
	r r
	r r
	r
	r
}
\toprule

\mr[c]{2}{*}{\textbf{Hs.}}
	& \mc{2}{c}{\textbf{Quantor}}
	& \mc{2}{c}{\textbf{Konjunktion}}
	& \mr[c]{2}{*}{\textbf{Summe}}
	& \mr[c]{2}{*}{\textbf{Wortformen}}
	\\

\cmidrule(rl){2-3}
\cmidrule(rl){4-5}

%
	& \textit{bėid(e)}
	& \textit{bėidiu}
	& \textit{bėid(e)}
	& \textit{bėidiu}
	\\

\midrule

\citet{kc:A1}
	& 24
	& 
	& 16
	& 21
	& 61
	& 90.068
	\\

\citet{kc:M}
	& 26
	& 
	& 
	& 40
	& 66
	& 84.658
	\\

\citet{kc:H}
	& 26
	& 
	& 37
	& 
	& 63
	& 86.573
	\\

\midrule

\citet{kc:B1}
	& 26
	&  6
	&  2
	& 28
	& 62
	& 79.850
	\\

\citet{kc:P}
	& 17
	& 
	& 16
	& 
	& 33
	& 28.717
	\\

\citet{kc:VB}
	& 26
	&  1
	& 25
	& 24
	& 76
	& 76.000
	\\

\midrule

\citet{kc:C1}
	& 13
	&  2
	& 
	& 34
	& 49
	& 87.791
	\\

\citet{kc:K}
	& 15
	&  2
	&  1
	& 32
	& 50
	& 89.888
	\\

\citet{kc:Z}
	& 17
	& 
	& 31
	& 
	& 48
	& 90.939
	\\

\midrule

Summe
	&     190
	&      11
	&     128
	&     179
	&     508
	& 714.484
	\\

\bottomrule
\end{tabular}
\label{tab:beidevar}
\end{table}

Trotz des Vorbehalts zur zahlenmäßigen Verteilung der Belege auf die
unterschiedlichen syntaktischen Kontexte ist gerade der Vergleich der
bairischen Handschrift \citet{kc:B1} zum Urkunden\-material interessant, da
\citet{kc:B1} die einzige der neun untersuchten Handschriften ist, in der im
Kontext des kombinierten gemischtgeschlechtlichen Bezugs beim Quantor
\norm{bėide} überhaupt beide Formen, \norm{bėide} und \norm{bėidiu}, auftreten.

In der Untersuchung des \CAO{} wird sich zeigen, dass in bairischen
Urkunden -- anders als in den meisten \KC{}-Handschriften -- durchaus
Variation zwischen diesen beiden Formen herrscht. Abgesehen von \norm{bėide} in
seiner Funktion als Quantor kommt Variation in der Funktion als Konjunktion bei
den Handschriften \citet{kc:A1} und \citet{kc:VB} vor. Auch \citet{kc:VB} ist
in Bezug auf das Urkundenmaterial insofern bemerkenswert, als
\citet[224]{schneider1987a} die Handschrift \citet{kc:VB} unter der Rubrik zum
letzten Viertel des 13.~Jahrhunderts aufführt; \citeauthor{wolf:kckat} schätzt
sie auf \blockcquote[65]{wolf:kckat}{um 1290/1300}. Die Handschrift dürfte also
etwa zeitgleich mit dem Großteil der Urkunden des \CAO{} entstanden
sein (vgl.~\cref{tab:urkstat}).

Generell stellt sich bei der Auswertung der \KC{}-Belege die Frage
nach deren Zählweise. Anders als beim \CAO{}, das nur wenige
Parallelausfertigungen enthält \autocite[vgl.][326--328]{ganslmayeretal2003},
verteilen sich 456 von insgesamt 508 aus der \KC{} exzerpierte
Targets auf 110 identifizierte Parallelstellen. Dies bedeutet,
dass 89,7\pct\ der Gesamtzahl der Belege mindestens einem anderen Parallelbeleg
zugeordnet werden können, wobei im Durchschnitt vier Belege ein solches Set aus
Parallelstellen ergeben, neben 52 Einzelbelegen.
% Wie können also die Beleghäufigkeiten in den Auswertungstabellen in
% \cref{ch:kcanalyse} angegeben werden -- einzeln nach absoluter Häufigkeit
% oder zusammengefasst nach Parallelstelle?
Da auch zwischen Parallelstellen mitunter Variation herrscht, würde eine
Zusammenfassung der zugehörigen Belege bei der Zählung die Statistik verzerren.
Jeden Beleg einzeln zu zählen erscheint daher als die bessere Strategie. Die
eher geringe Belegmenge pro Kontext erlaubt es in vielen Fällen, Einzelbelege
und ihre Parallelstellen anzugeben und gemeinsam zu diskutieren.

\subsection{Ausgeschlossene Belege}
\label{subsec:ausgeschlossene_kc}

In der \KC{} gibt es einzelne Stellen, die sich bei der Auswertung als
unverständlich oder fehlerhaft erwiesen und sich der morphologischen
Annotierung entzogen, oder aber in ungeeigneten Kontexten stehen und daher
ebenfalls nicht verwertbar sind. Darüber hinaus wurden die beiden
mitteldeutschen Handschriften \citet{kc:H} und \citet{kc:P} sowie die
schwäbische Handschrift \citet{kc:Z} nicht regulär in die Auswertung
einbezogen, da dort regelmäßig keine Variation zwischen \norm{bėide} und
\norm{bėidiu} vorliegt \autocite[vgl. auch][183]{ksw2}.
% % Belege aus diesen Handschriften werden gelegentlich aber zum Vergleich
% % mitangeführt.
% Im Folgenden werden nur diejenigen Belege in die Analyse einbezogen, die nicht
% aufgrund von fehlender Variation in der jeweiligen Handschrift ausgeschlossen
% wurden (siehe~\cref{sec:adjdeclkc}).

Der Fall eines vermuteten Abschreibfehlers wird durch den Beleg in
\cref{ex:kcexcl1} vertreten. Der Editionstext nach \nosh\citet{schroeder1895}
wird zum Vergleich in \cref{ex:kcexcl1_schroeder} angegeben. Die Korrektheit
der Transkription der Textstelle wurde am Digitalisat der Handschrift überprüft
und bestätigt.

\begin{exe}
\ex \begin{xlist}
	\ex \label{ex:kcexcl1}
		\gll von den pæpſten vnd von chûnigen \\
			von den Päpsten und von Königen \\
	\sn \gll Peide und frûmigen \\
			beide und tüchtigen \\
		\begin{taggedline}{\parencite[\pno~2\va, 20--21]{kc:B1}}
		\trans \wdef{von den Päpsten und von Königen, sowohl als auch tüchtigen
			\textins{sic}}
		\end{taggedline}

	\ex \label{ex:kcexcl1_schroeder}
		\gll von den bâbesen unt von den chunigen, \\
			von den Päpsten und von den Königen \\
	\sn \gll baidiu guoten unt ubelen, \\
			beide guten und bösen \\
		\begin{taggedline}{\parencite[19--20]{schroeder1895}}
		\trans \wdef{von den Päpsten und von den Königen, sowohl guten als auch
			bösen}
		\end{taggedline}
\end{xlist}
\end{exe}

Auch die älteste vollständige Handschrift \autocite{kc:A1} berichtet, die
Chronik handele von den Päpsten und von den Königen \lit{bædiv gvͦten vn̄ ubelen}
\wdef{sowohl guten als auch bösen} \autocites[1\ra,
18]{kc:A1}[vgl.][20]{schroeder1895}, insofern wird \norm{bėide} hier nicht als
nachgestellter Quantor zu werten sein. Die
B-\allowbreak{}Pa\-ral\-lel\-hand\-schrift \citet{kc:P} enthält hier die
Variante \lit{bejde boſen un̄ frumege} \wdef{sowohl bösen als auch
tüchtige\textins{n}} \autocite[\pno~1\ra, 10]{kc:P} und in der
B-Vergleichshandschrift \citet{kc:VB} lautet die Zeile \lit{Boͤſen vnd
frvͤmigen} \wdef{bösen und tüchtigen}, allerdings ohne einleitendes
\norm{bėide} \autocite[\pno~1\ra, 18]{kc:VB}. Am ehesten wird der Beleg in \cref{ex:kcexcl1} also
zur gleichen Gruppe wie seine Parallelstellen passen: \norm{bėide} als
korrelative Konjunktion mit zwei koordinierten Adjektiven (vgl.
\cref{subsec:beidkoordtarg}), von denen das erste ausfällt.

Mit Bezug auf \notecite[19--20]{schroeder1895} verweist \citet[26,
Anm.~45]{weis2022} mit \citet[55, Anm.~87]{dickhutbielsky2015} außerdem auf
eine Kontroverse um deren grammatische Bezüge. Ich stimme mit \citet{haupt2019}
überein, \notecite[20]{schroeder1895} \blockcquote[239]{haupt2019}{als Klammer
nur auf die zwei zunächst folgenden Wörter (\textit{guoten unt ubelen}), im
Sinne von \enquote{sowohl als auch}} bezogen zu lesen. Ob sich das Attribut
\lit{baidiu guoten unt ubelen} \wdef{sowohl guten als auch bösen} nur auf
die Könige oder auch auf die Päpste bezieht, lässt sich nicht eindeutig
beurteilen. Belege, die den Rückbezug der Konjunktion \lit{bėide} auf zwei
Konjunkte nahelegen, ließen sich im exzerpierten Material keine finden. Da
keine zwei Könige explizit genannt werden, erscheint eine Interpretation von
\lit{baidiu} als nachgestelltem Quantor mit Bezug auf \lit{chunigen}
\wdef{Königen} unplausibel, zumal in diesem Fall die Form des Quantors
(Nom./Akk.~Pl. N.) nicht kongruent mit dem Präpositionalobjekt wäre (Dat.~Pl.
M.).

% Zur Konstruktion \norm{bėidiu \dots\ unde} im Mittelhochdeutschen siehe
% \citet[626--628]{ksw2}. Eine allgemeine Diskussion zur Syntax korrelativer
% Konjunktionen aus Sicht der generativen Grammatik bietet
% \citet{johannessen2005}; aus Sicht der LFG behandelt
% \citet[366--368]{dalrymple2001} dieses Thema. Siehe jedoch auch
% \citet{peterson2004,borsley2005} mit gewichtigen Argumenten gegen
% \posscite{johannessen1998,johannessen2005} Modell von Koordination als
% endozentrische, das heißt hierarchische Konstruktion.

% % % Im weiteren Verlauf dieser Arbeit soll jedoch Hermeneutik weniger eine %
% % Rolle spielen; vielmehr soll die \KC{} als Textkorpus zur % %
% Untersuchung aus Perspektive der historischen Sprachwissenschaft dienen.

In einem anderen Fall bezieht sich in dem in \cref{ex:chindpaid_1} zitierten
Beleg \lit{chínd} \wdef{Kinder} auf die Knaben Faustinus und Faustus
\autocites[\pno~10\rb, 5--6]{kc:M}[vgl.][1239--1240]{schroeder1895}. Der Beleg
ist unflektiert. Die angegebenen Parallelstellen enthalten jeweils \lit{der
kinde baider} \wdef{der beiden Kinder}. Bei Voranstellung des schwach
flektierten Adjektivs wäre unabhängig vom Genus regulär mit \norm{bėiden} zu
rechnen, vergleiche \cref{ex:chindpaid_2}.

\begin{exe}
\ex \label{ex:chindpaid}
\begin{xlist}
	\ex \label{ex:chindpaid_1}
		\gll Frowe der trovm ergat dir niht zelaíd. \\
			Frau der Traum ergeht dir nicht zuleide \\
	\sn \gll Nv {vnder wínt} dich der chínd \textbf{paid}. \\
			nun nimm.an dich der Kind[\Gen.\Pl.\NeutM] beide[\Gen.\Pl.\NeutM] \\
		\begin{taggedline}{\parencites%
			[\pno~11\ra, 17]{kc:M}[vgl.]%
			[\pno~6\rb, 22--23]{kc:A1}
			[\pno~8\ra, 12--13]{kc:H}
			[1354]{schroeder1895}
		} % 1073x
		\trans \wdef{Frau, der Traum soll dich nicht belasten. Nun nimm dich der
			beiden Kinder an.}
		\end{taggedline}
		% → GEN.PL.M/F/N ansonsten stark -er, schwach -en.

	\ex \label{ex:chindpaid_2}
		\gll Gewan er {dreizich tvſent}. \\
			Gewann er dreißigtausend \\
	\sn \gll Aller \textbf{guten} chnehte. \\
			aller gut-\Gen.\Pl.\MascM.\Wk{} Knechte \\
		\begin{taggedline}{\parencites%
			[\pno~52\vb, 20--21]{kc:M}[vgl.]%
			[\pno~30\rb, 14--15]{kc:A1}
			[\pno~41\rb, 36--37]{kc:H}
			[\pno~19\vc, 23--24]{kc:B1}
			[\pno~33\vb, 20--21]{kc:VB}
			% [\pno~24\vc, 23--24]{kc:VC}
			[\pno~42\ra, 24--25]{kc:K}
			[\pno~137\va, 1--2]{kc:Z}
			[6967--6968]{schroeder1895}
		}
		\trans \wdef{gewann er \textins{=~Severus} für sich dreißigtausend
			von allen guten Soldaten.}
		\end{taggedline}
\end{xlist}
\end{exe}

Vergleichbare Belege mit Nachstellung des Adjektivs in einer determinierten NP
liegen in der Stichprobe lediglich vier vor. Davon enthält der einzige Beleg im
Akkusativ \cref{ex:maccadj_1} unflektiertes \lit{gut} \wdef{gut}, gegenüber
flektiertem \lit{guten} \wdef{guten} bei Voranstellung, wie exemplarisch in
\cref{ex:maccadj_2} gezeigt. Die Handschrift \citet{kc:A1} enthält an dieser
Stelle % [31\vb, 29]
tatsächlich \lit{manigen helt guͦten} \wdef{manch guten Helden}.

\begin{exe}
\ex \label{ex:maccadj}
	\begin{xlist}
	\ex \label{ex:maccadj_1}
		% \gll Der Hertzog von Meran. \\
		% 	der Herzog von Meran \\
		% \gll Vnd clauus ſeín man. \\
		% 	und \textins{Sc}lavus sein Mann \\
		\gll Di furten manigẽ helt \textbf{gut}. \\
			die führten viel-\Acc.\Sg.\MascM.\St{} Held gut[\Acc.\Sg.\MascM] \\
	\sn \gll Si chomen dar mít eínẽ mvt. \\
			sie kamen dahin mit vereintem Gesinnung \\
	\sn \gll Jn di ſtat zerome. \\
			in die Stadt zu=Rom \\
		\begin{taggedline}{\parencites%
			[\pno~55\vb, 20--24]{kc:M}[vgl.]%
			[\pno~31\vb, 27--31]{kc:A1}
			[\pno~43\vb,15--19]{kc:H}
			[\pno~38\va, 18--21]{kc:C1}
			% [\pno~26\rb, 23--27]{kc:VC}
			[\pno~44\rb, 28--32]{kc:K}
			[\pno~145\ra, 18--21]{kc:Z}
			[7346--7350]{schroeder1895}
		}
		\trans \wdef{%
			% Der Herzog von Meran und \textins{Sc}lavus, sein Mann,
			\textelp{} die führten viele gute Helden an. Sie kamen einmütig hin
			zu der Stadt Rom.}
		\end{taggedline}

	\ex \label{ex:maccadj_2}
		\gll Do vragt er di geſvnden \\
			da fragte er die Leute \\
	\sn \gll Wo er iheſum moht vínden. \\
			wo er Jesus könnte finden \\
	\sn \gll Den vil \textbf{guten} arzat \\
			den viel gut-\Acc.\Sg.\MascM.\Wk{} Arzt \\
		\begin{taggedline}{\parencites%
			[\pno~6\rb, 31]{kc:M}[vgl.]%
			[\pno~3\vb, 26--28]{kc:A1}
			[\pno~4\rb, 28--30]{kc:H}
			[\pno~4\rb, 35--36]{kc:B1}
			[\pno~7\ra, 9--10]{kc:P}
			[\pno~4\ra, 24--27]{kc:C1}
			[\pno~12\va, 12--15]{kc:Z}
			[723--725]{schroeder1895}
		}
		\trans \wdef{Da fragte er die Leute, wo er Jesus finden
			könnte, den sehr guten Arzt.}%
		%
		% 	\footnote{Die Handschriften variieren hier relativ stark im
		% 		Wortlaut, der Tenor ist jedoch, dass der Bote in der Stadt
		% 		herumfragt. Die beiden parallelen A-Handschriften enthalten
		% 		eine Formulierung mit \norm{gesinde} \wdef{Dienerschaft,
		% 		Leute}: \lit{doͮ frac er di geſinden} \autocite[\pno~3\vb, 26]{kc:A1};
		% 		\lit{Do uragete er daz geſinde} \autocite[\pno~4\rb, 28]{kc:H}
		% 		\wdef{Da fragte er die Leute}.
		% 		% Durch das Schlüsselwort
		% 		% \norm{arƶāt} \wdef{Arzt} beeinflusst könnte der Schreiber von
		% 		% \citet{kc:M} hier \norm{gesinde} mit \norm{gesunde}
		% 		% \wdef{Gesunde/r} verwechselt haben.
		% 		}
		\end{taggedline}
		\\
	\end{xlist}
\end{exe}

Auffällig an \cref{ex:maccadj_1} ist, dass \lit{gut(en)} \wdef{gut(en)} auf
\norm{mvt} \wdef{Gesinnung, Stimmung, Geist} gereimt wird. Es ist möglich,
dass an dieser Stelle die Endung aus Reimzwang wegfällt. Andererseits liegt in
der Adjektivstichprobe auch in anderen untersuchten Handschriften im Kontext
der Nachstellung eine Tendenz zur Endungslosigkeit vor. Dies wird bestätigt
durch die Beobachtung von \citet[241]{ksw2}, die allgemein eine über das
13.~Jahrhundert hin zunehmende Neigung zum Weglassen der Flexion beim
nachgestellten Adjektiv im Reim feststellen. Die Endungslosigkeit von
\lit{paid} in \cref{ex:chindpaid_1} wird vermutlich ebenfalls in diesem
Zusammenhang stehen.

\phantomsection
\label{phsec:kchiatus}
Entsprechend der von \citet{gjelsten1980} genannten methodischen Kritikpunkte
an \posscite{askedal1974} Studie wurden außerdem solche Belege nicht in die
Auswertung einbezogen, die vor Vokal oder in Reimposition stehen.
\citet{askedal1973} merkt bezüglich eines ähnlichen Falls in seiner Auswertung
der \citetitle{maroldschroeder1969}-Edition von \citet{maroldschroeder1969}
bereits selbst an, dass denkbar sei, dass die Flexionsendung \norm{iu} in
Hiatuspositionen bei der Rezitation apokopiert werde und möglicherweise
hyperkorrekt dafür \norm{e} geschrieben werde \autocite[90--91]{askedal1973}.

Des Weiteren führt er mit Bezug auf \citet[662--663]{grimm1870} die
Reimposition als für eine morphosyntaktische Analyse problematisch an, insofern
\blockcquote[89]{askedal1973}{die Flexionsendung \norm{-iu} des Nom.\ Sing.\
Fem.\ und Nom.\ Akk.\ Plur.\ Neutr.\ in der mhd.\ Epik niemals in der
Reimposition am Zeilenende vorkommt}. In der Belegsammlung zur \KC{}
finden sich insgesamt 133 Belege von 510 in Neutralisierungspositionen;
\cref{ex:neutralpos,ex:neutralpos2} geben Beispiele aus beiden Gruppen.
Gegenbeispiele, in denen \norm{bėidiu} in der Reimposition vorkommt, sind auch
im hier exzerpierten Material nicht vorhanden, allerdings liegen auch keine
Belege vor, in denen andernfalls regulär mit \norm{-iu} zu rechnen wäre. Belege
mit \norm{iu} vor Vokal gibt es immerhin elf, davon bis auf einen
\cref{ex:neutralpos2} nur bei \norm{bėide} als Konjunktion.

\begin{exe}
\ex \label{ex:neutralpos}
	\gll Den gewalt habten ſi beid\textbf{e} \\
		den Macht hatten \Tpl\subM.\Nom{} beide-\Nom.\Pl\subM.\St{} \\
\sn \gll Daz enſchol v̂ns nímmer werden leide \\
		das \Neg=wird uns nie werden Leid \\
	\begin{taggedline}{\parencites%
		[\pno~13\ra, 41--42]{kc:B1}[vgl.]%
		[\pno~18\rb, 3]{kc:A1}
		[\pno~24\vb, 35]{kc:H}
		[\pno~32\ra, 3]{kc:M}
		[\pno~21\ra, 31]{kc:VB}
		[\pno~37\ra, 26]{kc:P}
		[\pno~23\rb, 5]{kc:C1}
		[\pno~25\rb, 39]{kc:K}
		[\pno~83\ra, 5]{kc:Z}
		[4261--4264]{schroeder1895}
	}
	\trans \wdef{Diese Macht hatten sie beide. Das wird uns nie Leid werden.}
	\end{taggedline}

\ex \label{ex:neutralpos2}
	\begin{xlist}
	\ex \label{ex:neutralpos2_1}
		\gll der berch der haizet gargan. \\
			der Berg der heißt Gargân \\
	\sn \gll da worden ſi baid\textbf{e} \textbf{\sscr{o}{v}}ffe erſlagen. \\
			da wurden \Tpl\subM.\Nom{} beide-\Nom.\Pl.\MascM.\St{} auf
				erschlagen \\
		\begin{taggedline}{\parencites%
			[\pno~33\rb, 28--30]{kc:A1}[vgl.]%
			[\pno~46\ra, 12]{kc:H}
			[\pno~58\vb, 15]{kc:M}
			[\pno~40\va, 15]{kc:C1}
			[\pno~46\va, 30]{kc:K}
			[\pno~152\va, 5]{kc:Z}
			[7704--7705]{schroeder1895}}
		\trans \wdef{Der Berg heißt Gargân, darauf wurden sie beide erschlagen.}
		\end{taggedline}

	\ex \label{ex:neutralpos2_2}
		\gll Rome vnd Lateran \\
			Rom und Lateran \\
	\sn \gll Wurden ím baid\textbf{\sscr{u}{ı}} \textbf{v}ndertan \\
			wurden ihm beide-\Nom.\Pl.\NeutI.\St{} untertan \\
		\begin{taggedline}{\parencites%
			[\pno~68\vb, 13--14]{kc:K}[vgl.]%
			[\pno~49\vb, 15--16]{kc:A1}
			[\pno~69\ra, 29--30]{kc:H}
			[\pno~87\ra, 32--33]{kc:M}
			[\pno~31\vb, 18--19]{kc:B1}
			[\pno~82\va, 45--46]{kc:VB}
			[\pno~60\ra, 14--15]{kc:C1}
			% [\pno~39\vc, 25--26]{kc:VC}
			[\pno~229\ra, 6--7]{kc:Z}
			[11416--11417]{schroeder1895}}
		\trans \wdef{Rom und Lateran wurden ihm beide untertan.}
		\end{taggedline}
		\\
	\end{xlist}
\end{exe}

Mit Ausnahme der Handschrift \citet{kc:K} wurden keine alemannischen
\KC{}-Textzeugen in die Analyse aufgenommen. Der Vollständigkeit halber
seien in \cref{ex:beidkcalem} die wenigen Belege zu \norm{bėide} aus solchen
Fragmenten aufgeführt, die im \citet{hsc} als spezifisch alemannisch
ausgewiesen werden \autocite[vgl.][4, 44, 54]{wolf:kckat}.

\begin{exe}
\ex \label{ex:beidkcalem}
	\begin{xlist}
	\needspace{3\baselineskip}
	\ex \gll bediv rome vnde Lateran \\ % 2013
		     beide Rom[\Acc.\Sg.\NeutI] und Lateran[\Acc.\Sg.\MascI] \\
		\begin{taggedline}{\parencites[\pno~2\vb, 17]{kc:a11}[vgl.][5953]{schroeder1895}}
		\trans \wdef{sowohl Rom als auch der Lateran}
		\end{taggedline}

	\ex \gll die iuncherren beide \\ % 1091
		     die Junker-\Nom.\Pl{} beide-\Nom.\Pl.\MascM{} \\
		\begin{taggedline}{\parencites[\pno~1\vb, 14]{kc:a14}[vgl.][1417]{schroeder1895}}
		\trans \wdef{die beiden jungen Herren}
		\end{taggedline}

	\ex \gll Beidiv die ſtat \textins{\lit{u}}n̄ daz lan\textins{\lit{t}} \\ % Ø
		     beide die Stadt[\Acc.\Sg.\FemI] und das Land[\Acc.\Sg.\NeutI] \\
		\begin{taggedline}{\parencite[\pno~1\rb, 4]{kc:b1}}
		\trans \wdef{sowohl die Stadt als auch das Land}
		\end{taggedline}

	\ex \gll Beidiv wip un̄ man \\ % 2004
		     beide Frau[\Nom.\Sg.\NeutF] und Mann[\Nom.\Sg.\MascM] \\
		\begin{taggedline}{\parencites[\pno~2\rb, 29]{kc:b1}[vgl.][619]{schroeder1895}}
		\trans \wdef{Frau wie Mann}
		\end{taggedline}
\end{xlist}
\end{exe}
