\addchap{\lsAbbreviationsTitle}

\section{Allgemeine Abkürzungen}

\noindent%
\begin{tabularx}{\linewidth}{@{} l X @{}}
Bz.			& Bezirk (Österreich) \\
Dépt.		& Département (Frankreich) \\
Kr.			& Kreis (Deutschland) \\
XP			& Konstituente vom Typ \emph{X} \\
\xbar{X}	& Projektion eines Kopfes vom Typ \emph{X} \\
\xhead{X}	& Kopf vom Typ \emph{X} \\
\end{tabularx}

%%%%%%%%%%%%%%%%%%%%%%%%%%%%%%%%%%%%%%%%%%%%%%%%%%%%%%%%%%%%%%%%%%%%%%%%%%%%%%%%

\section{Abkürzungen der Annotation}

Die grammatische Annotation von Beispielen richtet sich nach den
\citetitle{lgr} \autocite{lgr}. Darüber hinaus werden die folgenden Abkürzungen
verwendet:\\

\noindent%
\begin{tabularx}{\linewidth}{@{} >{\scshape}l X @{}}
\SM		& männlich \\
\SF		& weiblich \\
\SX		& blah \\
\SA		& blah \\
case	& Kasus \\
cl		& Nominalklasse \\
concord	& grammatischer Konkord innerhalb der Nominalphrase \\
deg		& Grad \\
gend	& Genus \\
index	& Index der anaphorischen Referenz \\
obj		& Objekt \\
preconj	& Präkonjunktion \\
sex		& Sexus \\
st		& stark \\
subj	& Subjekt \\
wk		& schwach \\
\end{tabularx}

%%%%%%%%%%%%%%%%%%%%%%%%%%%%%%%%%%%%%%%%%%%%%%%%%%%%%%%%%%%%%%%%%%%%%%%%%%%%%%%%

\section{Abgekürzte Literatur}

\noindent%
\begin{tabularx}{\linewidth}{@{} >{\itshape}l >{\raggedright\arraybackslash}X @{}}
\CAO	& \tit{Corpus der altdeutschen Originalurkunden bis zum Jahr 1300}
		\autocite{} \\
\KC		& \tit{Kaiserchronik} \autocite{} \\
\REA	& \citetitle{ddd} \autocite{ddd} \\
\REM	& \citetitle{rem} \autocite{rem} \\
ZfdA	& Zeitschrift für deutsches Altertum und deutsche Literatur \\
\end{tabularx}

\section{Handschriften der \tit{Kaiserchronik}}
\label{sec:hssverzkc}

Die folgenden Handschriften und Fragmente der \tit{Kaiserchronik} werden mit
ihren Siglen im Text genannt. Weiterführende Informationen zur Handschrift
sowie Links zu Digitalisaten können dem \tit{Handschriftencensus} (\tit{HSC};
\nosh\cite{hsc}) unter der jeweiligen Handschriften-ID entnommen werden.%
%
	\footnote{Siehe z.\,B.~\url{https://handschriftencensus.de/1432} für A1.}
%
\\

\noindent%
\begin{tabularx}{\linewidth}{@{} l X l @{}}
A1	& Vorau, Archiv des Augustiner-Chorherrenstifts, Ms 276
	& \autocite[1432]{hsc} \\
a11	& Nürnberg, Germanisches Nationalmuseum, Hs. 22067
	& \autocite[1189]{hsc} \\
a14	& Straßburg, Bibliothèque nationale et universitaire, ms. 2215
	& \autocite[1828]{hsc} \\
B1	& Wien, Österreichische Nationalbibliothek, Cod. 2779
	& \autocite[2693]{hsc} \\
b1	& Basel, Universitätsbibliothek, Cod. N I 3, Nr. 89
	& \autocite[1158]{hsc} \\
C1	& Wien, Österreichische Nationalbibliothek, Cod. 2685
	& \autocite[2013]{hsc} \\
H	& Heidelberg, Universitätsbibliothek, Cod. pal. germ. 361
	& \autocite[1181]{hsc} \\
K	& Karlsruhe, Badische Landesbibliothek, Cod. Aug. 52
	& \autocite[8470]{hsc} \\
M	& München, Bayerische Staatsbibliothek, Cgm 37
	& \autocite[2119]{hsc} \\
P	& Prag, Národní knihovna České republiky, Cod. XIII.G.43
	& \autocite[1168]{hsc} \\
T	& München, Bayerische Staatsbibliothek, Cgm 965
	& \autocite[8472]{hsc} \\
VB	& Wien, Österreichische Nationalbibliothek, Cod. 2693
	& \autocite[1215]{hsc} \\
VC	& Wien, Österreichische Nationalbibliothek, Cod. 12487
	& \autocite[3394]{hsc} \\
W	& Wolfenbüttel, Herzog August Bibliothek, Cod. Guelf. 15.2 Aug. 2º
	& \autocite[6668]{hsc} \\
Z	& Leutkirch, Fürstliche Waldburg zu Zeil und Trauch\-burg\-sches Gesamt\-archiv, ZAMs 30
	& \autocite[8471]{hsc} \\
\end{tabularx}
