\addchap{\lsPrefaceTitle}
 
Ein herzliches Dankeschön gilt allen, die in den letzten fünf Jahren mit ihrem
Zuspruch und ihrer Unterstützung dazu beigetragen haben, dass diese Arbeit ein
gutes Ende gefunden hat. An erster Stelle sind das meine Betreuer Jürg
Fleischer und Jürgen Wolf. Von euch durfte ich lernen, philologisches und
linguistisches Know-how gewinnbringend miteinander zu verknüpfen. Danke, dass
ihr mir den Freiraum gegeben habt, bei dieser Arbeit alle Register meiner
Interessen zu ziehen, stets ein offenes Ohr für Fragen hattet und nie verlegen
wart, euch Zeit zur Korrektur und Besprechung von Entwürfen zu nehmen.

Oliver Schallert danke ich für den Funken, der letztlich die Idee zu dieser
Untersuchung gezündet hat. Danke auch für deine Gastfreundschaft, sei es in
Marburg oder in Augsburg, sowie für dein freundschaftliches Mentoring seit es
mich als studentische Hilfskraft in der Deutschen Philologie des Mittelalters
vor gut zehn Jahren zwecks Kooperation zwischen unseren beiden Abteilungen in
die Sprachgeschichte verschlagen hat. Deine konstruktive Kritik an dieser
Arbeit hat dazu beigetragen, sie noch ein Stückchen besser zu machen.

Dank gilt darüber hinaus Daniel David Weis, Lea Schäfer und Hanna Fischer. Auch
von euren Vorschlägen, Korrekturen, Nachfragen und Anmerkungen durfte ich
profitieren. Ungenauigkeiten und Fehler, die jetzt noch im Text enthalten sind,
liegen allein in meiner Verantwortung. Daniel David Weis und Magnus Breder
Birkenes möchte ich außerdem herzlich danken für viele gute Gespräche und
freundschaftliche Ermutigungen, durchzuhalten, ganz besonders während zäher
Zeiten.

\fw{Last but not least} danke ich meinen Eltern Ingolf und Brita. Ohne eure
Ermutigung und Unterstützung von Kindheit an, meiner Neugier, meinen Interessen
und meinen Talenten zu folgen, wäre diese Arbeit nie geschrieben worden.

\begin{flushright}
Carsten Becker\\
Marburg im Mai 2022
\end{flushright}

 